% Elektrotechnische Grundlagen
%  » In welchen Fällen wir uns damit beschäftigen müssen und in welchen nicht (vorgefertigte vs. individuelle Hardware)
%  » Was ist Strom? Spannung vs. Stromstärke, Wechselstrom vs. Gleichstrom
%  » Elementare Bauteile: Stromquellen, Wiederstände, Kondensatoren, Dioden, Transistoren, Quarzkristalle (Symbol, Funktionsweise, Beispiel)
%  » Digitale (logische) Schaltungen vs. analoge Schaltungen, Tri-State-Ausgänge
%  » Logikelemente / Logikfamilien: AND, OR, XOR, NOT, NAND (Symbol, Funktionsweise, Beispiel)
%  » Vorgehen beim Entwurf einer einfachen Schaltung (Pull-Up, Pull-Down, …)

% Ausgewählte Hardwareschnittstellen im Detail (jeweils mit technischer Vorstellung, typischen Bauteilen und einer Beispielschaltung)
%  » Digitale Ein- und Ausgänge via GPIO (Beispiel: Taster, Beispiel: Keyboard-Matrix)
%  » Analoge Ein- und Ausgänge (low-res, nicht Audio, Beispiel: Wiederstandsleiter)
%  » Pulsweitenmodulation (Beispiel: LED-Helligkeit oder Elektromotor)
%  » Serial Peripherial Interface
%  » I²C
%  » Fallbeispiel mit mehreren Komponenten

% Vorgehen bei der Entwicklung
% (anhand eines kleinen Fallbeispiels)
% (Anmerkung: Der Ablauf ist nicht streng-linear)
%  » Definition der Anforderungen
%  » Anfertigen einer groben Hardwareskizze
%  » Bauelemente suchen (Mouser, Digikey, …)
%  » Datenblätter studieren (was steht drin?)
%  » Schaltplan ausarbeiten
%  » Firmware entwickeln
%  » Schaltung auf Breadboard/Lochrasterplatine testen
%  » PCB-Layout anfertigen
%  » PCB fertigen lassen

% Ausblick: FPGA, Verilog/VHDL (nur erwähnen, dass es das gibt)

% Hausaufgabe: Youtube-Videos (anschauen und Fragen dazu beantworten)
%  » ??? Irgendwas elementares zu Widerständen und co. ???
%  » Ben Eater: Tri-State Logic ???
%  » Ben Eater: Nachbau einfacher Logikelemente mit Transistoren ???
%  » Ben Eater: Wie ein Computer addiert ???
%  » ??? Irgendwas zum Vorgehen beim Hardwareentwurf ???

% Hausaufgabe: Bauteile recherchieren (welche?)
%  » Suche nach Luftqualitätssensoren
%  » Studium des Datenblatts (grundlegende Informationen herauslesen)

% Übungsstunde: Entwurf einer Alarmanlage
%  » Bewegungssensor zur Aktivierung der Alarmanlage
%  » LCD-Display zur Ausgabe einer Eingabeaufforderung
%  » Knöpfe zur Eingabe des Deaktivierungscodes
%  » Kamera zum Aufnahmen von Fotos
%  » Vorgegebenes NodeRED-Programm zum Auslösen eines Alarams per E-Mail


% Quellen
% https://de.m.wikipedia.org/wiki/Tri-State
%
% https://htl.moedling.at/fileadmin/_migrated/content_uploads/Grundlagen_der_ET_02.pdf
% Zwar "All rights reserved", ggf. aber auch einfache und anschauliche Erklärungen
%
% https://www.grund-wissen.de/
% Enthält Skripte zu Eletronik, Linux, C und Python
%
% https://www.grund-wissen.de/elektronik/_downloads/grundwissen-elektronik.pdf
% CC-BY-3.0 Skript zu den Grundlagen der Elektronik
%
% https://www.gut-erklaert.de/physik/grundlagen-eletrotechnik.html
% Absolut minimalistische Einführung, ggf. als leichte Einstiegslektüre geeignet
%
% https://www.elektronik-kompendium.de/sites/grd/index.htm
% Umfangreiche Artikelsammlung zu verschiedenen Grundlagen der Elektronik
%
% https://www.elektronik-kompendium.de/
% Weitere Artikel der gleichen Serie
%
% https://www.elektronik-kompendium.de/shop/buecher/elektronik-fibel
% https://www.elektronik-kompendium.de/shop/buecher/formelbuch
% Buchempfehlungen

\begin{frame}{Lernziele}
    \begin{itemize}
        \item Lernziel 1
        \item Lernziel 2
        \item Lernziel 3
        \item Lernziel 4
    \end{itemize}
\end{frame}

%-------------------------------------------------------------------------------
\section{Abschnitt}
%-------------------------------------------------------------------------------

\begin{frame}{Folie}
    Inhalt der Folie
\end{frame}
