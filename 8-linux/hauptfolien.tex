%%% Folie
\begin{frame}{Ausgangslage}
    Hier die Ausgangslage/Motivation für das Kapitel beschreiben ...

    % UNGEFÄHRE AUSSSAGE
    %
    % Während der Entwicklung oder auch bei sehr kleinen Installationen ist es kein
    % Problem, die Konfiguration des Raspberry Pi direkt am Gerät anzupassen und dabei
    % alle benötigten Programme von Hand zu installieren. Jedoch müssen auch dann
    % Vorkehrungen getroffen werden, damit die benötigten Dienste zum Hochfahren des
    % Pi automatisch starten. In größeren Setups ist dieses Vorgehen jedoch unzweckmäßig,
    % da ein defektes Gerät nicht einfach ausgetauscht oder weitere Geräte angebunden
    % werden können. Dieses Kapitel soll daher nicht nur zeigen, wie einfache Anpassungen
    % direkt in Raspbian gemacht werden können, sondern auch wie ein komplett eigenes
    % Linux-Image erstellt werden kann, das lediglich auf eine SD-Karte geflasht werden
    % muss, um weitere Geräte in Betrieb nehmen zu können.
\end{frame}

%%% Folie
\begin{frame}{Lernziele}
    \begin{block}{Linux auf dem Raspberry Pi}
        \begin{itemize}
            \item Linux im Vergleich zu anderen Betriebssystemen einordnen können
            \item Den typischen Aufbau eines Linux-Systems beschreiben können
            \item Den Filesystem Hierarchy Standard kennen und verstehen
            \item Den Bootvorgang des Raspberry Pi vollständig erklären können
            \item Raspbian für eigene kleine Projekte konfigurieren können
        \end{itemize}
    \end{block}

    \begin{block}{Erstellung eigener Linux-Systeme}
        \begin{itemize}
            \item Die Vorgehensweise beim Bau eigener Images kennen und verstehen
            \item Ein geeignetes Toolkit zum Bau von Linux-Systemen auswählen können
            \item Eigene Linux Images mit den vorgestellten Toolkits erzeugen können
            \item Eigene Programme in die selbst erstellten Images integrieren können
        \end{itemize}
    \end{block}
\end{frame}

%%% Folie
\begin{frame}{Inhaltsübersicht}
    \tikzset{
        MyTitle/.style={
            align=left,
            text=blue!65!purple,    % 65% blue, 35% purple
            anchor=north west,
            font=\large,
        },
        MyNode1/.style={
            below=2pt,
            align=left,
            anchor=north west,
            font=\footnotesize
        },
        MyNode2/.style={
            above=2pt,
            align=left,
            anchor=south west,
            font=\footnotesize
        },
        MyNumber/.style={
            text=white,
            font=\scriptsize
        }
    }

    % Vgl. https://tex.stackexchange.com/a/18201
    \pgfdeclarelayer{bg}
    \pgfsetlayers{bg, main}

    \begin{columns}
        \column{\dimexpr\paperwidth-1.4cm}
        \begin{tikzpicture}
            % Linux auf dem Rasbperry Pi
            \node[MyTitle] at (0,0) {Linux auf dem Rasbperry Pi};
            \filldraw[fill=darkred]
                ( 0,-0.6) circle (4pt) node[MyNumber](pi-1){1}  node[MyNode1] {Anforderungen an \\ eingebettete Betriebssysteme}
                ( 4,-0.6) circle (4pt) node[MyNumber](pi-2){2}  node[MyNode1] {Linux im Vergleich zu \\ anderen Betriebssystemen}
                ( 8,-0.6) circle (4pt) node[MyNumber](pi-3){3}  node[MyNode1] {Typischer Aufbau \\ eines Linux-Systems};
            \filldraw[fill=gray!50]
                (11,-0.6) circle (4pt) node[](pi-tr){}
                (11,-2.6) circle (4pt) node[](pi-br){};
            \filldraw[fill=darkred]
                ( 8,-2.6) circle (4pt) node[MyNumber](pi-4){4}  node[MyNode2] {Der Bootvorgang des \\ Raspberry Pi im Detail}
                ( 4,-2.6) circle (4pt) node[MyNumber](pi-5){5}  node[MyNode2] {Benutzer- und Netzwerk- \\ konfiguration in Raspbian}
                ( 0,-2.6) circle (4pt) node[MyNumber](pi-6){6}  node[MyNode2] {Konfiguration des \\ Raspbian-Startvorgangs};

            % Erstellung eigener Linux-Systeme
            \node[MyTitle] at (0,-3.4) {Erstellung eigener Linux-Systeme};
            \filldraw[fill=darkred]
                ( 0,-4.0) circle (4pt) node[MyNumber](mk-1){7}  node[MyNode1] {Generelles Vorgehen beim \\ Bau eines Firmare-Images}
                ( 4,-4.0) circle (4pt) node[MyNumber](mk-2){8}  node[MyNode1] {Vergleich verschiedener \\ Baukästen für Linux}
                ( 8,-4.0) circle (4pt) node[MyNumber](mk-3){9}  node[MyNode1] {Linux Images bauen \\ mit Rasbperry Pi Gen};
            \filldraw[fill=gray!50]
                (11,-4.0) circle (4pt) node[](mk-tr){}
                (11,-6.0) circle (4pt) node[](mk-br){};
            \filldraw[fill=darkred]
                ( 8,-6.0) circle (4pt) node[MyNumber](mk-4){10} node[MyNode2] {Linux Images bauen \\ mit Buildroot}
                ( 4,-6.0) circle (4pt) node[MyNumber](mk-5){11} node[MyNode2] {Linux Images bauen \\ mit Ubuntu Core}
                ( 0,-6.0) circle (4pt) node[MyNumber](mk-6){12} node[MyNode2] {Fazit und Ausblick};

            % Verbindungslinien
            \begin{pgfonlayer}{bg}
                \draw (pi-1) -- (pi-2) -- (pi-3) -- (pi-tr) -- (pi-br) -- (pi-4) -- (pi-5) -- (pi-6);
                \draw[gray, densely dashed] (pi-6) -- (mk-1);
                \draw (mk-1) -- (mk-2) -- (mk-3) -- (mk-tr) -- (mk-br) -- (mk-4) -- (mk-5) -- (mk-6);
            \end{pgfonlayer}
        \end{tikzpicture}
    \end{columns}
\end{frame}

%-------------------------------------------------------------------------------
\section{Linux auf dem Raspberry Pi}
%-------------------------------------------------------------------------------

%%% Folie
\begin{frame}{Folie}
    TODO
\end{frame}

%%% Folie
\begin{frame}{Folie}
    TODO
\end{frame}

%%% Folie
\begin{frame}{Folie}
    TODO
\end{frame}

%-------------------------------------------------------------------------------
\section{Erstellung eigener Linux-Systeme}
%-------------------------------------------------------------------------------

%%% Folie
\begin{frame}{Folie}
    TODO
\end{frame}

%%% Folie
\begin{frame}{Folie}
    TODO
\end{frame}

