%%% Folie
\begin{frame}{Ausgangslage}
    Hier die Ausgangslage/Motivation für das Kapitel beschreiben ...

    % UNGEFÄHRE AUSSSAGE
    %
    % Während der Entwicklung oder auch bei sehr kleinen Installationen ist es kein
    % Problem, die Konfiguration des Raspberry Pi direkt am Gerät anzupassen und dabei
    % alle benötigten Programme von Hand zu installieren. Jedoch müssen auch dann
    % Vorkehrungen getroffen werden, damit die benötigten Dienste zum Hochfahren des
    % Pi automatisch starten. In größeren Setups ist dieses Vorgehen jedoch unzweckmäßig,
    % da ein defektes Gerät nicht einfach ausgetauscht oder weitere Geräte angebunden
    % werden können. Dieses Kapitel soll daher nicht nur zeigen, wie einfache Anpassungen
    % direkt in Raspbian gemacht werden können, sondern auch wie ein komplett eigenes
    % Linux-Image erstellt werden kann, das lediglich auf eine SD-Karte geflasht werden
    % muss, um weitere Geräte in Betrieb nehmen zu können.
\end{frame}

%%% Folie
\begin{frame}{Lernziele}
    \begin{itemize}
        \item Den Bootvorgang des Raspberry PI kennen und verstehen
        \item Die wichtigsten Verzeichnisse des Linux FSH kennen und verstehen
        \item Eigene einfache SystemD-Servicedefinitionen schreiben können
        \item Eigene Linux Images auf Basis von Ubuntu Core erzeugen können
        \item Eigene Python-Programme in Snaps für Ubuntu Core verpacken können
        %\item Erklären können, wie mit Buildroot einfache Linux Images gebaut werden
        %\item Begründete Auswahl zwischen Buildroot und Ubuntu Core treffen können
    \end{itemize}
\end{frame}


%-------------------------------------------------------------------------------
\section{Abschnitt}
%-------------------------------------------------------------------------------

%%% Folie
\begin{frame}{Folie}
    TODO
\end{frame}

%%% Folie
\begin{frame}{Folie}
    TODO
\end{frame}

%%% Folie
\begin{frame}{Folie}
    TODO
\end{frame}

%-------------------------------------------------------------------------------
\section{Abschnitt}
%-------------------------------------------------------------------------------

%%% Folie
\begin{frame}{Folie}
    TODO
\end{frame}

%%% Folie
\begin{frame}{Folie}
    TODO
\end{frame}

