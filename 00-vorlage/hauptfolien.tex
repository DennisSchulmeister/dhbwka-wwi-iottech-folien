\begin{frame}{Lernziele}
    \begin{itemize}
        \item Die generelle Struktur der Foliensätze kennen
        \item Die wichtigsten \latex-Befehle anwenden könnnen
        \item Beamer- und Handout-Dateien bauen können
        \item Sich freuen \smiley
    \end{itemize}
\end{frame}

%-------------------------------------------------------------------------------
\section{Beispiele zu \latex Beamer}
%-------------------------------------------------------------------------------

\begin{frame}{Nur Text}
    Dies ist eine einfache Folie mit nur Text.
\end{frame}

\begin{frame}{Aufzählung}
    \begin{itemize}
        \item Punkt 1
        \item Punkt 2
        \item Punkt 3

        \pause
        \item Dieser Punkt kommt erst nach einer Pause.
    \end{itemize}
\end{frame}

\begin{frame}{Blöcke}
    \begin{block}{Einfacher Block}
       Dies ist eine wichtige Information.
    \end{block}

    \begin{alertblock}{Alert-Block}
        Dies ist eine "Warnung" oder so etwas in der Art …
    \end{alertblock}

    \begin{exampleblock}{Beispiel-Block}
        Dies ist ein Beispiel.
    \end{exampleblock}

    \begin{definition}{Definition}
        Dies ist eine Definition.
    \end{definition}
\end{frame}

\begin{frame}{Spalten}
    \begin{columns}
        \column{.5\textwidth}
        \begin{center}
            Spalte 1 mit \texttt{\\column}
        \end{center}

        \column{.5\textwidth}
        \begin{center}
            Spalte 2 mit \texttt{\\column}
        \end{center}
    \end{columns}

    \bigskip

    \begin{columns}
        \begin{column}{.5\textwidth}
            \begin{center}
                Spalte 1 mit \texttt{\\begin\{column\}}
            \end{center}
        \end{column}
        \begin{column}{.5\textwidth}
            \begin{center}
                Spalte 2 mit \texttt{\\begin\{column\}}
            \end{center}
        \end{column}

    \end{columns}
\end{frame}

\begin{frame}{Automatisch aufklappende Liste}
    \begin{enumerate}[<+->]
        \item Punkt 1
        \item Punkt 2
        \item Punkt 3
        \item Punkt 4
    \end{enumerate}
\end{frame}

\begin{frame}{Overlay-Spezifikationen}
    Das Aussehen einer Folie kann stufenweise verändert werden, um einzelne
    Punkte stellenweise hervorzuheben bzw. ein- und auszublenden. Dies erfolgt
    mit den sog. \textcolor<2->{MidnightBlue}{"Overlay Spezifikationen"}.

    \bigskip

    \uncover<3->{Spezifikationen werden als \texttt{<2-3>} an verschiedene
    Makros angehängt. \texttt{<2-3>} bedeutet, dass der entsprechende Inhalt
    nur in den Schritten 2 bis 3 ausgewertet wird.}

    \bigskip

    \uncover<4->{Der Von- oder Bis-Bereich ist optional, z.B. \texttt{<2->}}
\end{frame}

\begin{frame}{Overlay-Spezifikationen}
    \textbf{Makros, die Spezifikationen unterstützen:}

    \begin{itemize}[<+->]
        \item \texttt{\Slash uncover<>\{\}}:
        Einblenden mit zuvor reserviertem Platz

        \item \texttt{\Slash only<>\{\}}:
        Einblenden ohne zuvor reserviertem Platz

        \item \texttt{\Slash item<>\{\}}:
        Einblenden von Listenpunkten

        \item \texttt{\Slash alt<>\{\}\{\}}:
        Erstes Argument bei \texttt{true}, sonst zweites

        \item \texttt{\Slash textbf<>\{\}}, \texttt{\Slash textit<>\{\}}, \ldots:
        Fettdruck, Kursivschrift, \ldots

        \item \texttt{\Slash textcolor<>\{farbe\}\{\}}:
        Einfärbung eines Textabschnitts

        \item \texttt{\Slash alert<>\{\}}:
        Rote Hervorhebung

        \item \texttt{\Slash begin\{\}}:
        Die meisten Umgebungen
    \end{itemize}
\end{frame}

\begin{frame}{Animierte Tabellen}
    \begin{tabularx}{\textwidth}{|X|X|X|}
        \hline
        \textbf{Eisoma} & \textbf{Standort} & \textbf{Anbindung} \\
        \hline

        \uncover<2->{Stammfiliale} & \uncover<2->{Grünwinkel} & \uncover<2->{Auto oder Fahrrad} \\
        \hline

        \uncover<3->{Forchheim} & \uncover<3->{Hauptstraße} & \uncover<3->{Zu Fuß} \\
        \hline

        \uncover<4->{Mörsch} & \uncover<4->{Rösselsbrünnle} & \uncover<4->{Zu Fuß} \\
        \hline
    \end{tabularx}
\end{frame}

\begin{frame}
    \FullscreenImage{img/screenshot.png}
\end{frame}

\begin{frame}{Java-Code}
    \lstinputlisting[language=Java]{code/gpio.java}
\end{frame}
