%%% Folie
\begin{frame}{Ausgangslage}
    Hier die Ausgangslage/Motivation für das Kapitel beschreiben ...

    % UNGEFÄHRE AUSSAGE:
    %
    % Im dritten Semester haben wir bereits typische Hardwareschaltungen für
    % einfache Embedded-Anwendungen kennengelernt. Zwar haben wir dabei mit dem
    % DHT11 auch schon einen häufig genutzten Umweltsensor integriert, die meisten
    % Bauteile waren aber nur simple analoge Bauteile wie Schalter, LEDs oder Relais.
    % Für komplexere Anwendungen steht darüber hinaus eine Vielzahl an Sensoren und
    % Aktoren zur Verfügung, die über die unterschiedlichsten Hardwareschnittstellen
    % angebunden werden müssen. Dieses Kapitel soll daher über das letzte Semester
    % hinaus einen Überblick über die im IoT-Kontext wichtigsten Schnittstellen und
    %deren Programmierung in Python geben.
\end{frame}

%%% Folie
\begin{frame}{Lernziele}
    \begin{itemize}
        \item Funktionsweise serieller Schnittstellen erklären können
        \item Synchrone und asynchrone, serielle Schnittstellen abgrenzen können
        \item Bausteine mit SPI und I²C mit dem Raspberry Pi verbinden können
        \item SPI und I²C-Bausteine mit Python ansprechen können
        %\item Einzelbilder und Videos mit der PiCamera in Python aufnehmen können
        %\item Die Daten eines GPS-Receiver in Python auslesen können
    \end{itemize}
\end{frame}


%-------------------------------------------------------------------------------
\section{Einführung in das vierte Semester}
%-------------------------------------------------------------------------------

%%% Folie
\begin{frame}{Folie}
    TODO
\end{frame}

%%% Folie
\begin{frame}{Folie}
    TODO
\end{frame}

%%% Folie
\begin{frame}{Folie}
    TODO
\end{frame}

%-------------------------------------------------------------------------------
\section{GPIO-Programmierung (Wiederholung)}
%-------------------------------------------------------------------------------

%%% Folie
\begin{frame}{Folie}
    TODO
\end{frame}

%%% Folie
\begin{frame}{Folie}
    TODO
\end{frame}

%-------------------------------------------------------------------------------
\section{Serielle Kommunikation}
%-------------------------------------------------------------------------------

%%% Folie
\begin{frame}{Folie}
    TODO
\end{frame}

%%% Folie
\begin{frame}{Folie}
    TODO
\end{frame}
