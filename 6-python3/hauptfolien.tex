%%% Folie
\begin{frame}{Ausgangslage}
    Hier die Ausgangslage/Motivation für das Kapitel beschreiben ...

    % UNGEFÄHRE AUSSAGE
    %
    % Die einfachen Programmbeispiele der vorangegangen Kapitel eignen sich in der
    % Regel kaum für produktive Anwendungen, da sie einem einfachen, prozeduralen
    % Programmierstil gehorchen, bei dem jeder Methodenaufruf den rufenden Thread
    % so lange blockiert, bis die gewünschte Aktion durchgeführt wurde. Auf diese
    % Weise lassen sich daher nur sehr wenige Sensoren und Aktoren verbinden und
    % bestimmte Anwendungsfälle, wie die Bereitstellung einer deviceseitigen REST-API
    % lassen sich auf diese Weise gar nicht realisieren. Darüber hinaus wird der
    % Quellcode auch schnell sehr komplex und unübersichtlich, je mehr Komponenten
    % integriert werden sollen. Dieses Kapitel soll daher die wesentlichen
    % Architekturmuster zeigen, mit denen sich komplexe Pythonprogramme in lose
    % gekoppelte Einheiten zerlegen lassen, die eine einfache Erweiterbarkeit und
    % Wartbarkeit des Quellcodes (z.B. wenn eine Komponente ausgetauscht werden
    % muss) ermöglichen.
\end{frame}

%%% Folie
\begin{frame}{Lernziele}
    \begin{itemize}
        \item Komplexe Anwendungsfälle objektorientiert modellieren können
        \item Python-Anwendungen mit Klassen und Modulen modularisieren können
        \item Die Nachteile einer zu engen Kopplung von Klassen verstehen können
        \item Entwurfsmuster Observer und Message Broker anwenden können
        \item Nebenläufige Programmierung mit Threads in Python
        %\item Nebenläufige Programmierung mit Coroutinen in Python
    \end{itemize}
\end{frame}


%-------------------------------------------------------------------------------
\section{Abschnitt}
%-------------------------------------------------------------------------------

%%% Folie
\begin{frame}{Folie}
    TODO
\end{frame}

%%% Folie
\begin{frame}{Folie}
    TODO
\end{frame}

%%% Folie
\begin{frame}{Folie}
    TODO
\end{frame}

%-------------------------------------------------------------------------------
\section{Abschnitt}
%-------------------------------------------------------------------------------

%%% Folie
\begin{frame}{Folie}
    TODO
\end{frame}

%%% Folie
\begin{frame}{Folie}
    TODO
\end{frame}

