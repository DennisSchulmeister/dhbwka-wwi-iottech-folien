\lstdefinelanguage{JavaScript}{
  keywords={break, case, catch, class, const, constructor, continue, debugger, default, delete, do, else, finally, for, function, if, in, instanceof, let, new, return, super, switch, this, throw, try, typeof, var, void, while, with},
  morecomment=[l]{//},
  morecomment=[s]{/*}{*/},
  morestring=[b]',
  morestring=[b]",
  sensitive=true
}

%%% Folie
\begin{frame}{Lernziele}
    \begin{itemize}
        \item Kennenlernen von Python und  Einsatzmöglichkeiten
        \item Programmiercode strukturieren und Module verwenden
        \item Messwerte auf dem Raspberry Pi auslesen
        \item Grundlegender Umgang mit Visual Studio Code
        \item Umgang mit Problemstellungen beim Programmieren für IoT Devices
        \item Wichtige Programmierbausteine kennenlernen
    \end{itemize}
\end{frame}

%-------------------------------------------------------------------------------
\section{Allgemeines zu Python}
%-------------------------------------------------------------------------------

%%% Folie
\begin{frame}{Warum Python?}
    \begin{figure}[!htb]
        \includegraphics[scale=0.25]{3-python1/img/240px-Python-logo-notext}
    \end{figure}

    \begin{itemize}
        \item Plattformunabhängig  $\rightarrow$ interpretierte Sprache
        \item Arm an Boiler-Plate Code
        \item Knapp gehaltene Syntax
        \item Open Source
        \item Unterstützt verschiedene Paradigmen
        \item Viele Software-Bibliotheken vorhanden
    \end{itemize}
\end{frame}

%%% Folie
\begin{frame}{Wo wird Python eingesetzt?}
    \begin{itemize}
        \setlength{\itemindent}{3.0in}
        \item [\textbf{ Laut Python Package Index (siehe \url{https://pypi.org})}]
    \end{itemize}

    \begin{itemize}
        \item Rapid Prototyping
        \item Administratives Scripting
        \item Wissenschaftliche Software/Data Science
        \item Machine Learning und Big Data
        \item ...
     \end{itemize}
\end{frame}

%%% Folie
\begin{frame}{Python und Raspberry Pi}
    \begin{itemize}
        \setlength{\itemindent}{.3in}
        \item [\textbf{ Fakten}]
    \end{itemize}

    \begin{itemize}
        \item Historisch steht Pi für Python Interpreter
        \item Ursprünglich sollte in Python programmiert werden
        \item Python Interpreter ist üblicherweise vorinstalliert in Raspbian
     \end{itemize}
\end{frame}

%%% Folie
\begin{frame}{Python 2 vs. Python 3}
    \begin{itemize}
        \setlength{\itemindent}{1.0in}
        \item [\textbf{ Warum wichtig ?}]
    \end{itemize}

    \begin{itemize}
        \item Viele Beispiele und Libraries verwenden noch Version 2
        \item Version 2 wird ab 2020 nicht mehr offiziell von den Entwicklern unterstützt
        \item Die größeren bekannten und wichtigsten Projekte setzen auf Python 3
        \item Siehe \url{https://python3statement.org}
        \item Zum Aufrufen des Interpreters sollte man daher python3 nutzen
        \item Vergleich und Tipps: \url{https://devopedia.org/python-2-vs-3}
     \end{itemize}
\end{frame}

%%% Folie
\begin{frame}{Entwicklung}
    \begin{itemize}
        \setlength{\itemindent}{2.1in}
        \item [\textbf{ Entwicklung mit Visual Studio Code }]
    \end{itemize}

    \begin{itemize}
        \item Leichtgewichtiger Editor mit vielen Plugins
        \item Python Plugin für Interpreter
        \item Installation auf dem Pi möglich,  siehe: \url{https://developer-blog.net/visual-studio-code-auf-dem-raspberry-pi/}
        \item Auf eigenen Notebook möglich:, Versionen für Windows/Linux/Mac, siehe: \url{https://code.visualstudio.com/Download}
        \item Remote arbeiten vom eigenen Notebook auf dem Pi möglich, siehe: \url{https://www.hanselman.com/blog/VisualStudioCodeRemoteDevelopmentOverSSHToARaspberryPiIsButter.aspx}
     \end{itemize}
\end{frame}


%-------------------------------------------------------------------------------
\section{Python anhand einfacher Beispiele}
%-------------------------------------------------------------------------------

%%% Folie
\begin{frame}[fragile]{Python vs. Java vs. JavaScript - Hello World}
    \begin{lstlisting}[language=Python, gobble=8]
        # Hello World in Python
        print('Hello World!')
    \end{lstlisting}
    \begin{lstlisting}[language=JavaScript, gobble=8]
        // Hello World in Java Script (Backend Version)
        console.log('Hello World!')
    \end{lstlisting}
    \begin{lstlisting}[language=Java, gobble=8]
        /* Hello World in Java */
        public class HelloWorld {
            public static void main(String[] args) {
                System.out.println("Hello World");
            }
        }
    \end{lstlisting}
\end{frame}

%%% Folie
\begin{frame}[fragile]{Python vs. Java vs. JavaScript - Funktionen}
    \begin{lstlisting}[language=Python, gobble=8]
        # Funktion in Python
        def my_function(parameter):                 # keine Typeninformation
            a = 1
            return parameter * parameter + a
    \end{lstlisting}
    \begin{lstlisting}[language=JavaScript, gobble=8]
        // Funktion in Java Script
        function myFunction(parameter){             // keine Typeninformation
            let a = 1;
            return parameter * parameter + a;
        }
    \end{lstlisting}
    \begin{lstlisting}[language=Java, gobble=8]
        /* Methode in Java */
        public int myFunction (int parameter){      // Typeninformation
            int a = 1;
            return parameter * parameter + a;
        }
        \end{lstlisting}
\end{frame}

%%% Folie
\begin{frame}[fragile]{Python - Type Hints}
    Weitere Infos siehe \url{https://docs.python.org/3/library/typing.html}

    \begin{lstlisting}[language=Python, gobble=8]
        # Funktion zu Berechnung
        def my_function(parameter: int) -> int:         # Hinweise zu Typen
            a = 1
            return parameter * parameter + a
    \end{lstlisting}
	
	\begin{lstlisting}[language=Python, gobble=8]
        # Funktion zur Stringprüfung
        def my_other_function(parameter: str) -> bool:  # Hinweise zu Typen
            return len(parameter) > 0
    \end{lstlisting}
   
\end{frame}

%%% Folie
\begin{frame}[fragile]{Python - Lambda Funktion}

    \begin{lstlisting}[language=Python, gobble=8]
        # Funktion in Python
        def my_function(parameter: int) -> int:
            a = 1
            return parameter * parameter + a
    \end{lstlisting}
	
    \begin{lstlisting}[language=Python, gobble=8]
        # lambda in Python
        my_function = lambda x: x * x + 1 # Kurzform der Funktion als Lambda
    \end{lstlisting}
   
\end{frame}

%%% Folie
\begin{frame}[fragile]{Python vs. Java vs. JavaScript - Klassen}
    \begin{lstlisting}[language=Python, gobble=8]
        class MyClass(Superklasse):                     # Python
            __init__(self, value):
                super().__init__(value)
                self.value = value;

            def my_method(self):
                return self.value
    \end{lstlisting}
    \begin{lstlisting}[language=JavaScript, gobble=8]
        class MyClass extends Superklasse{              // JavaScript
            constructor() {
                super(value);
                this.value = value;
            }
            myMethod() {
                return this.value;
            }
        }
    \end{lstlisting}
    \begin{lstlisting}[language=Java, gobble=8]
        public class MyClass extends Superklasse{       // Java
            private int value;

            MyClass(int value) {
                super(value);
                this.value = value;
            }
            public int myMethod () {
                return this.value;
            }
        }
    \end{lstlisting}
\end{frame}

%%% Folie
\begin{frame}[fragile]{Python Duck Typing 1}
    Beispiel basierend auf \url{https://hackernoon.com/python-duck-typing-or-automatic-interfaces-73988ec9037f}

    \begin{lstlisting}[language=Python, gobble=8]
        """
        Intention: ein Vogel gibt Laut, Ergebnis: Ein Betrüger geht als Vogel
        durch, denn er kann ja quack machen und nimmt das Geld mit.
        """

        class Ente:
            def quack(self):
                print('Quack!')

        class Gans:
            def quack(self):
                print('Quaaaack')

        class Quacksalber:
            def quack(self):
                print('Quack')
                print('Ich nehme dann mal dein Geld!')

        list = [Ente(), Gans(), Quacksalber()]

        for item in list:
            item.quack()
    \end{lstlisting}
\end{frame}

%%% Folie
\begin{frame}[fragile]{Python Duck Typing 2}
    \begin{lstlisting}[language=Python, gobble=8]
        """
        Intention: Verschiedene Sensoren sollen über gemeinsame Schnittstelle
        die Distanz zurückliefern
        """

        class UltraschallSensor:
            def get_distance(self):
                # ... über Bibliothek Auslesen, bspw. 200 cm
                return input() # zum Test: User Eingabe über Keyboard

        class InfrarotSensor:
            def get_distance(self):
                return  input()

        def print_distance(sensor):
            print(sensor.get_distance())

        print_distance(UltraschallSensor())
    \end{lstlisting}
\end{frame}

%%% Folie
\begin{frame}[fragile]{Python Datenstrukuren 1}
    \begin{lstlisting}[language=Python, gobble=8]
        """
        Intention: Dictionary mit vordefinierten und erweiterbaren Schlüsseln,
        Format ähnlich zu Java Script Object Literals
        """

        # Meet the Dictionary
        switcher = {
            15: 'Ganz nah',     # Schlüssel : Wert
            30: 'Nah',
            70: 'Weit weg'
        }
        switcher[99] = 'Gerade noch messbar'

        ergebnis = switcher.get(15, 'UNGUELTIG')

        # Ähnlich wie in JavaScript kann das Dictionary (hier switcher genannt)
        # auch Funktionsobjekte beinhalten. Dies wird auf folgender Seite erklärt:
        # https://jaxenter.com/implement-switch-case-statement-python-138315.html

        # Analog in if-else:
        if(eingabe_schuessel == 15):
            # tue etwas
            pass
        elif(eingabe_schuessel == 30):
            # tue etwas anderes
            pass
        else:
            # z.B. ungueltig ausgeben
            pass
    \end{lstlisting}
\end{frame}

%%% Folie
\begin{frame}[fragile]{Python Datenstrukuren 2}
    \begin{lstlisting}[language=Python, gobble=8]
        """
        Intention: Eine Liste mit mehreren Zahlen, welche die Helligkeit einer
        Leuchtdiode darstellen sollen
        """

        # Meet the List
        intensities = [0, 0.5, 1]

        print(intensities[2])

        for value in intensities:
            print(value)

        intensities.append(1.5)
        intensities.remove(0)
        intensities.pop(0)
    \end{lstlisting}
\end{frame}

%%% Folie
\begin{frame}[fragile]{Python Datenstrukuren 3}
    \begin{lstlisting}[language=Python, gobble=8]
        """
        Intention: Zusammengehörige Werte im Code zusammenhalten
        """

        # Meet the Tuple
        sensor_messung = (0.3, '2019-10-07-15:00:01.996')

        print('Der Messwert um ', sensor_messung[1], ' war: ', sensor_messung[0])
    \end{lstlisting}

    \bigskip
    \bigskip

    \parbox{\linewidth}{
        \textbf{Anmerkung:} Anders als eine Liste kann ein Tuple nachträglich nicht mehr
        verändert werden. Es können daher keinen neuen Werte einfügt oder alte
        Werte entfernt werden. Auch können die enthaltenen Werte nicht überschrieben
        werden. Man sagt, ein Tupel ist „unveränderlich” (immutable).
    }
\end{frame}

%%% Folie
\begin{frame}[fragile]{Sauberes Programmieren: Coding Conventions}
    \begin{itemize}
    \setlength{\itemindent}{.5in}
     \item [\textbf{ Richtlinien}]
    \end{itemize}
    \begin{itemize}
        \item Python Enhancement Proposals enthalten u.a. Guidelines für lesbaren Code
        \item Siehe: \url{https://pep8.org/#naming-conventions}
        \item Autopep hilfreich für die Entwicklungsumgebung, siehe:  \url{https://pypi.org/project/autopep8/}
     \end{itemize}
\end{frame}

%%% Folie
\begin{frame}[fragile]{Sauberes Programmieren: Zen of Python}
    \url{https://www.python.org/dev/peps/pep-0020/#the-zen-of-python}

    \begin{lstlisting}[language=Python, gobble=8]
        """
        Zen: japanische Richtung des Buddhismus, die durch Meditation die Erfahrung der
        Einheit allen Seins und damit tätige Lebenskraft und größte Selbstbeherrschung
        zu erreichen sucht.
        """

        # Python Easter Egg:
        import this
    \end{lstlisting}

    \begin{Verbatim}[fontsize=\scriptsize, gobble=8]
        Beautiful is better than ugly.
        Explicit is better than implicit.
        Simple is better than complex.
        Complex is better than complicated.
        Flat is better than nested.
        Sparse is better than dense.
        Readability counts.
        Special cases aren't special enough to break the rules.
        Although practicality beats purity.
        Errors should never pass silently.
        Unless explicitly silenced.
        In the face of ambiguity, refuse the temptation to guess.
        There should be one-- and preferably only one --obvious way to do it.
        Although that way may not be obvious at first unless you're Dutch.
        Now is better than never.
        Although never is often better than *right* now.
        If the implementation is hard to explain, it's a bad idea.
        If the implementation is easy to explain, it may be a good idea.
        Namespaces are one honking great idea -- let's do more of those!
    \end{Verbatim}
\end{frame}


%-------------------------------------------------------------------------------
\section{Python auf dem Raspberry Pi}
%-------------------------------------------------------------------------------

%%% Folie
\begin{frame}{Beispiel: Sensoren auslesen mit GPIO 1}
    \begin{itemize}
    \setlength{\itemindent}{.4in}
     \item [\textbf{ Ziele}]
    \end{itemize}
    \begin{itemize}
        \item  LED mittels PWM ansteuern
        \item  Abstandsmessung einbinden
        \item  Abstandsmessung und LED verbinden
        \item  Code modular aufbauen
        \item  Module zusammensetzen
     \end{itemize}
\end{frame}

%%% Folie
\begin{frame}{Beispiel: Sensoren auslesen mit GPIO 2}
    \begin{itemize}
    \setlength{\itemindent}{1in}
     \item [\textbf{ Wo anfangen?}]
    \end{itemize}
    \begin{itemize}
        \item  Python verwendet Bibliotheken, die man als Module einbindet
        \item  GPIO bzw. GPIOZero
        \item  Die Module werden installiert über:  pip3 install -r requirements.txt
        \item  requirements.txt: Auflistung der Libraries, aehnlich zu package.json in nodeJS oder build.gradle oder pom.xml in Java
     \end{itemize}
\end{frame}

%%% Folie
\begin{frame}[fragile]{Beispiel: Sensoren auslesen mit GPIO 3}
    \begin{lstlisting}[language=Python, gobble=8]
        #Inhalt: requirements.txt
        numpy==1.17.2
        schedule==0.6.0
        pypubsub==4.0.3
    \end{lstlisting}
\end{frame}

%%% Folie
\begin{frame}{Beispiel: Sensoren auslesen mit GPIO 4}
    \begin{itemize}
    \setlength{\itemindent}{1.0in}
     \item [\textbf{ Dokumentation}]
    \end{itemize}

    \begin{itemize}
        \item Datenblatt der Sensoren: \url{https://produktinfo.conrad.com/datenblaetter/1400000-1499999/001413759-an-01-de-SENSORKIT_X40_FUER_EINPLATINEN_COMPUTER.pdf}
        \item API zu GPIOZero: \url{https://gpiozero.readthedocs.io/en/stable/}
     \end{itemize}
\end{frame}

%%% Folie
\begin{frame}[fragile]{Python Sourcecode Struktur: Hauptprogramm}
    \begin{lstlisting}[language=Python, gobble=8]
        """ Basiert auf:  https://gpiozero.readthedocs.io/en/stable/recipes.html """
        import sys                                    # Systemfunktionen
        import logging                                # Ausgabe von Nachrichten

        from gpiozero import PWMLED                   # Bibliothek für PWM-Ansteuerung
        from time import sleep                        # Funktion zum Pausieren

        logger = logging.getLogger(__name__)          # Log-Ausgabe nicht mit print()
        led = PWMLED(17)                              # LED Kontrolle für GPIO PIN 17

        def led_dim(value):
            led.value = value                         # LED auf Wert setzen
            sleep(1)                                  # 1 Sekunde warten

        try:
            while True:                               # Programm am Laufen halten
                led_dim(0)                            # LED aus
                led_dim(0.5)                          # LED 50% Helligkeit
                led_dim(1)                            # LED 100% Helligkeit
        except KeyboardInterrupt:
            logger.info('Ende wegen User Eingabe')
        except:
            logger.error('Ende wegen Programmfehler: %s', sys.exc_info()[0])
        finally:
            led.close()                               # GPIOs zurücksetzen
    \end{lstlisting}
\end{frame}

%%% Folie
\begin{frame}[fragile]{Python Sourcecode Struktur: Modul 1}
    \begin{lstlisting}[language=Python, gobble=8]
        import sys                                    # Systemfunktionen
        import logging                                # Ausgabe von Nachrichten

        from gpiozero import PWMLED                   # Bibliothek für PWM-Ansteuerung
        from time import sleep                        # Funktion zum Pausieren

        class LedBlinker:
            __led = PWMLED(17)                        # LED Kontrolle für GPIO PIN 17
            __logger = logging.getLogger(__name__)    # Log-Ausgabe nicht mit print()

            def led_dim(self, value):
                self.__led.value = value              # LED auf Wert setzen
                sleep(1)                              # 1 Sekunde warten

            def blink_led(self):
                try:
                    for duration in [0, 0.5,1, 0.5]:  # aus, 50%, 100%, 50%
                        self.led_dim(duration)        # LED dimmen
                except KeyboardInterrupt:
                    __logger.info('Ende wegen User Eingabe')
                except:
                    # Siehe https://docs.python.org/3/tutorial/errors.html
                    __logger.error('Programmfehler: %s', sys.exc_info()[0])
                finally:
                    self.__led.close()                # GPIOs zurücksetzen
    \end{lstlisting}
\end{frame}

%%% Folie
\begin{frame}[fragile]{Python Sourcecode Struktur: Modul 1 einbinden}
    \begin{lstlisting}[language=Python, gobble=8]
        import logging                                # Ausgabe von Nachrichten
        from led_util import LedBlinker               # Klasse LedBlinker

        logger = logging.getLogger(__name__)          # Log-Ausgabe nicht mit print()

        if __name__ == '__main__':
            blinker = LedBlinker()

            try:
                while True:                           # Programm am Laufen halten
                    blinker.blink_led()               # LED blinken lassen
            except KeyboardInterrupt:
                logger.info('Ende wegen User Eingabe')
    \end{lstlisting}
\end{frame}

%%% Folie
\begin{frame}[fragile]{Python Sourcecode Struktur: Modul 2}
    \begin{lstlisting}[language=Python, gobble=8]
        import sys                                    # Systemfunktionen
        import logging                                # Ausgabe von Nachrichten+
        from gpiozero import DistanceSensor           # Bibliothek für Ultraschallsensor

        class Ultraschallsensor:
            #  Sensor über GPIO 17 und 18 ansteuern
            __sensor = DistanceSensor(echo=18, trigger=17, threshold_distance = 0.1, max_distance=1)
            __logger = logging.getLogger(__name__)    # Log-Ausgabe nicht mit print()

            def __init__(self):
                pass

            def read_distance_ic_cm(self):
                distance = -1

                try:
                    distance = self.__sensor.distance * 100
                except KeyboardInterrupt:
                    self.__logger.info('Ende wegen User Eingabe')
                except:
                    # Siehe https://docs.python.org/3/tutorial/errors.html
                    self.__logger.error('Programmfehler: %s', sys.exc_info()[0])
                finally:
                    self.__sensor.close() # PIN Konfiguration zurücksetzen

              return distance

            # __sensor.when_out_of_range = handle_out_of_range
            # Funktion, die z.B. loggt
    \end{lstlisting}
\end{frame}

%%% Folie
\begin{frame}[fragile]{Python Sourcecode Struktur: Modul 1  und 2 einbinden}
    \begin{lstlisting}[language=Python, gobble=8]
        import logging                                # Ausgabe von Nachrichten

        from led_util import LedBlinker               # LedBlinker-Klasse
        from distance_util import Ultraschallsensor   # Ultraschallsensor-Klasse

        __logger = logging.getLogger(__name__)        # Log-Ausgabe nicht mit print()

        if __name__ == '__main__':
            blinker = LedBlinker()
            ultraschall_sensor = Ultraschallsensor()

            try:
                while True:                           # Programm am Laufen halten
                    distance_in_cm = ultraschall_sensor.read_distance_in_cm()

                    if(distance_in_cm >= 15):
                        blinker.blink_led()  # LED
                    else:
                        logger.info('Nichts ungewoehnliches entdeckt')
            except KeyboardInterrupt:
                logger.info('Ende wegen User Eingabe')
    \end{lstlisting}
\end{frame}
