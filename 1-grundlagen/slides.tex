% Über die Vorlesung
%  » Vorstellung der Dozenten
%  » Vorstellung der Teilnehmer (Vorwissen und Grund der Modulwahl)
%  » Kompetenzziele laut Modulbeschreibung
%  » Inhalte der Vorlesung
%  » Vorausgesetztes Wissen?
%  » Ablauf des Semesters (9 Termine je Semester, Übungsstunden im Labor wg. den Bildschirmen)
%     1) Vorlesung: Grundlagen des Internet of Things
%     2) Vorlesung: Hardwaredesign für IoT-Anwendungen
%     3) Übungsstunde: Hardwaredesign (anhand vorgegebener NodeRED-Programme)
%     4) Übungsstunde: Hardwaredesign (anhand vorgegebener NodeRED-Programme)
%     5) Vorlesung: IoT-Anwendungen mit NodeRED
%     6) Vorlesung: Hardwarenahe Programmierung
%     7) Übungsstunde: NodeRED und hardwarenahe Programmierung
%     8) Übungsstunde: NodeRED und hardwarenahe Programmierung
%     9) Übungsstunde: NodeRED und hardwarenahe Programmierung
%  » Benötigte Software (Fritzing, NetBeans, Maven)
%  » Prüfungsform

% Anwendungsfälle
%  » Frage an die Studierenden: Was bedeutet IoT?
%  » Anwendungsfälle zuhause (Smart Home, Smart TV, Iot-Seminar: Honey Pi / Smart Mirror / Home Security)
%  » Anwendungsfälle in der Industrie (Losgröße 1, Predictive Maintenance, Edge Computing, was noch?)

% Geschichte des IoT
%  » Begriffsherkunft
%  » Apollo Guidance Computer (erstes Embedded System)
%  » Messen, Steuern, Regeln mit PDP11, CP/M, ...
%  » Vom Mainframe zum heutigen Single Board Computer (Miniaturisierung, Moorse Law, …)

% Architektur heutiger IoT-Anwendungen
% (Visualisieren mit einer Skizze, Jede Schicht kann mit jeder Schicht kombiniert werden)
%  » Hardwareplattformen / Development Boards (Pi, Espruino, ESP32, Pic, …)
%  » Hardwareschnittstellen (GPIO, SPI, I²C, RS-232, DSI/CSI, USB, WiFi, Bluetooth, ZigBee, …)
%  » Betriebssysteme (Linux, BSD, FreeRTOS, ITRON, Windows IoT, Android IoT, …)
%  » Programmiersprachen und Bibliotheken (Java, Python, JavaScript, …)
%  » Protokolle (IPv6, 6LoWPAN, MQTT, HTTP/REST, CoAP)
%  » Backendtechnologien (Gateway-Server, Microservices, Datenbanken, …)
%  » Benutzerschnittstellen (Webanwendungen, Mobile Apps, Desktop Client)

% In der Vorlesung verwendete Architektur:
% Raspberry Pi + Linux + Python/Java/JS??? + MQTT/CoAP + Selbstprogrammiertes Backend
% Was schauen wir uns in welchem Semester an?

% Embedded Hardware
%  » Abgrenzung: Prozessor vs. Microcontroller vs. System-on-Chip
%  » Grundlegende Hardwarearchitektur (Netzteil, Kristall, CPU, ROM/RAM, Peripherie)
%  » Programmierung: Assembler vs. Hochsprache (was wird wann genutzt?)
%  » Beispiel: Z80 basierte Architektur oder Intel MCS-51 oder Intel 8086 (Prozessor)
%  » Beispiel: AVR basierte Architektur (Microcontroller, z.B. Arduino)
%  » Beispiel: ARM basierte Architektur (System-on-Chip, z.B. Rasbperry Pi)

% Embedded Software
%  » Anforderungen an eingebettete Betriebssysteme (knappe Ressourcen, Non-Standard Hardware, Harte Stromtrennung, Echtzeit, …)
%  » Harte Echtzeit vs. Weiche Echtzeit
%  » Typischer Software Stack eines IoT-Devices

% Hausaufgabe: Fragen zu Linux
%  » Initial RAM Disk und PID 0 (anhand Folie 8 aus dem IoT-Workshop)
%  » File System Hierarchy Standard (anhand Folie 10 aus dem IoT-Workshop)
%  » Kernel vs. Userland
%  » Typische Linux-Distributionen ?
%  » Paketverwaltung unter Debian

% Keine Übungsstunde


\begin{frame}{Lernziele}
    \begin{itemize}
        \item Lernziel 1
        \item Lernziel 2
        \item Lernziel 3
        \item Lernziel 4
    \end{itemize}
\end{frame}

%-------------------------------------------------------------------------------
\section{Abschnitt}
%-------------------------------------------------------------------------------

\begin{frame}{Folie}
    Inhalt der Folie
\end{frame}
