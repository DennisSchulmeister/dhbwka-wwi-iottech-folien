%===============================================================================
\AufgabenHeader
%===============================================================================

%-------------------------------------------------------------------------------
\aufgabe{Eingebettete Systeme und IoT-Devices}
%-------------------------------------------------------------------------------
\teilaufgabe
Welche der folgenden Kriterien treffen auf eingebettete Systeme zu und unterscheiden
sie somit von konventionellen Computern?

\newcommand{\DefinitionEmbedded}[1]{
    {
        \renewcommand{\arraystretch}{1.5}
        \begin{longtable}{|p{0.71\textwidth}|p{0.1\textwidth}|p{0.1\textwidth}|}
            \hline
            \textbf{Kriterium} & \textbf{Wahr} & \textbf{Falsch} \\
            \endfirsthead

            \hline
            \textbf{Kriterium} & \textbf{Wahr} & \textbf{Falsch} \\
            \endhead

            \hline
            \endlastfoot

            \hline %1
            Es muss sich um vollwertige Computersysteme mit klar erkennbaren
            Ein- und Ausgabegeräten wie zum Beispiel einem Touchscreen
            oder einer Sprachausgabe handeln.
            &
            & #1
            \\

            \hline %2
            Es muss sich zumindest um einfache Systeme mit den Grundkomponenten
            eines Computers handeln. Es müssen daher mindestens ein Mikroprozessor
            oder ein Mikrocontroller, Speicher für den Programmcode, Hauptspeicher
            für temporäre Variablen sowie irgend eine Form der Ein- und Ausgabe
            vorhanden sein.
            & #1
            &
            \\

            \hline %3
            Die Computerarchitektur sollte möglichst allgemeingültig ausgelegt
            werden, um möglichst viele Anwendungsfälle abdecken zu können.
            &
            & #1
            \\

            \hline %4
            Das Computersystem sollte als solches vom Benutzer nicht wahrnehmbar
            sein, da es als Teil eines größeren Geräts verbaut wurde. Dies ist
            jedoch kein Muss, so dass eingebettete Systeme auch als eigenständige
            Geräte in Erscheinung treten können.
            &
            & #1
            \\

            \hline %5
            Kosten, Platzbedarf und Energieverbrauch müssen in der Regel optimiert
            werden, da die meisten eingebetteten Systeme in hoher Stückzahl produziert
            werden und oft auch unter eingeschränkten Bedingungen zuverlässig
            funktionieren müssen.
            & #1
            &
            \\

            \hline %6
            Es sollte ein möglichst nicht-deterministisches Systemverhalten
            angestrebt werden, um jegliche Unsicherheiten in der Reaktion auf
            Umwelteinflüsse zu minimieren.
            &
            & #1
            \\
        \end{longtable}
    }
}
\DefinitionEmbedded{}

\teilaufgabe
Welche der folgenden Kriterien muss ein eingebettetes System zusätzlich erfüllen,
damit es als IoT-Device bezeichnet werden kann?

\newcommand{\DefinitionIoT}[1]{
    {
        \renewcommand{\arraystretch}{1.5}
        \begin{longtable}{|p{0.71\textwidth}|p{0.1\textwidth}|p{0.1\textwidth}|}
            \hline
            \textbf{Kriterium} & \textbf{Wahr} & \textbf{Falsch} \\
            \endfirsthead

            \hline
            \textbf{Kriterium} & \textbf{Wahr} & \textbf{Falsch} \\
            \endhead

            \hline
            \endlastfoot

            \hline %1
            IoT-Devices müssen online oder offline beispielsweise über eine
            serielle Verbindung oder ein externes Speichermedium mit anderen
            Computern Daten austauschen.
            &
            & #1
            \\

            \hline %2
            IoT-Devices müssen zumindest zeitweise über eine Verbindung mit
            dem Internet besitzen, um mit anderen IoT-Devices und/oder einem
            oder mehrerer Backendservices zu kommunizieren.
            & #1
            &
            \\

            \hline %3
            IoT-Devices müssen nach heutiger Definition über ihre IP-Adresse
            eindeutig identifiziert oder zumindest für andere Geräte über das
            Internet erreichbar sein.
            & #1
            &
            \\

            \hline %4
            IoT-Devices benötigen zwingend ein Backend zur Registrierung,
            Überwachung und Verwaltung der Devices, da es sich sonst nur
            um eingebettete Systeme mit Internetverbidung handeln würde.
            &
            & #1
            \\

            \hline %5
            IoT-Devices treten meist in Form größerer Geräte in Erscheinung,
            die sich entweder über das Internet steuern lassen oder im weitesten
            Sinne mit Hilfe von Sensoren Informationen über ihre Umwelt sammeln
            und über das Internet bereitstellen.
            & #1
            &
            \\
        \end{longtable}
    }
}
\DefinitionIoT{}

\teilaufgabe
Welche besonderen Eigenschaften haben sogenannte \glqq Deeply Embedded Systems\grqq gegenüber
anderen eingebetteten Systemen oder IoT-Devices?

\begin{enumerate}
    %1
    \item Es handelt sich um kleinste Computersysteme mit in der Regel sehr
    geringer Komplexität und minimaler Systemausstattung.

    %2
    \item Es handelt sich stets um vernetzte Systeme, die mit anderen eingebetteten
    Systemen oder dem Internet kommunizieren.

    %3
    \item Es handelt sich um besonders unauffällig arbeitende Computersysteme,
    deren Vorhandensein für den Benutzer oft nicht offensichtlich ist.

    %4
    \item Es handelt sich um an extreme Bedingungen angepasste Computersysteme,
    die auch in besonderen Tiefen unter Wasser einsetzbar sind.
\end{enumerate}

\teilaufgabe
Welche Themengebiete sind beim Entwurf einer IoT-Architektur bestehend aus einem
oder mehrerer IoT-Devices, Backendservices und Benutzerschnittstelle zu beachten?
Nennen Sie die neun wesentlichen Bereiche und arbeiten sie sich dabei von unten
nach oben in Richtung Endbenutzer vor.

\bigskip
\teilaufgabe
An die Software eines eingebetteten Computersystems werden häufig besondere
Anforderungen gestellt. Um welche der folgenden Anforderungen handelt es sich
dabei? Streichen Sie alle nicht zutreffenden Aussagen.

\begin{longtable}{|p{0.45\textwidth} c p{0.45\textwidth}|}
    \hline
    Sparsame Ressourcennutzung
    & \textit{vs.} &
    Großzügige Zuteilung der Systemressourcen
    \\

    \hline
    Unterstützung üblicher Consumer-Hardware
    & \textit{vs.} &
    Unterstützung sehr spezialisierter Hardware
    \\

    \hline
    Geringe bis mittlere Nutzungsdauer am Tag
    & \textit{vs.} &
    Störungsfreier Betrieb über lange Zeit
    \\

    \hline
    Softwareupdates während dem Betrieb
    & \textit{vs.} &
    Updates in der Fachwerkstatt des Herstellers
    \\

    \hline
    Maximal flexible Systemarchitektur
    & \textit{vs.} &
    Vollständig deterministisches Systemverhalten
    \\

    \hline
    Nahezu unbegrenztes Multi-Tasking
    & \textit{vs.} &
    Feste Anzahl gleichzeitig laufender Tasks
    \\

    \hline
    Meist geringe Echtzeitanforderungen
    & \textit{vs.} &
    Erhöhte oder gar harte Echtzeitanforderungen
    \\

    \hline
    Ausschalten durch Trennen vom Strom
    & \textit{vs.} &
    Geordnetes Herunterfahren des Systems
    \\

    \hline
\end{longtable}

%-------------------------------------------------------------------------------
\aufgabe{Rechnerarchitekturen im Vergleich}
%-------------------------------------------------------------------------------
\teilaufgabe
TODO: Verschiedene Prozessoren einordnen

\teilaufgabe
TODO: Fragen zu einer Hardwareskizze

%-------------------------------------------------------------------------------
\aufgabe{Grundlagen der Assemblerprogrammierung}
%-------------------------------------------------------------------------------
TODO

%-------------------------------------------------------------------------------
\aufgabe{Komplexe vs. reduzierte Befehlssätze}
%-------------------------------------------------------------------------------
TODO

%-------------------------------------------------------------------------------
\aufgabe{Typischer Systemaufbau eingebetteter Systeme}
%-------------------------------------------------------------------------------
TODO

%===============================================================================
\clearpage
\LoesungHeader
%===============================================================================

%-------------------------------------------------------------------------------
\loesung{Eingebettete Systeme und IoT-Devices}
%-------------------------------------------------------------------------------
\teilaufgabe
Kriterien zur Definition von eingebetteter Systeme:
\DefinitionEmbedded{X}

\teilaufgabe
Kriterien zur Definition von IoT-Devices:
\DefinitionIoT{X}

\teilaufgabe
\glqq Deeply Embedded Systems\grqq im Vergleich zu anderen eingebetteten Systemen:

\begin{enumerate}
    \item Es handelt sich um kleinste Computersysteme mit in der Regel sehr
    geringer Komplexität und minimaler Systemausstattung.

    \setcounter{enumi}{2}   % Neuer Wert - 1
    \item Es handelt sich um besonders unauffällig arbeitende Computersysteme,
    deren Vorhandensein für den Benutzer oft nicht offensichtlich ist.
\end{enumerate}

\teilaufgabe
Themengebiete beim Entwurf einer IoT-Architektur (von unten nach oben):

\begin{enumerate}
    \item Elektronik
    \item Rechnerarchitektur
    \item Hardwareschnittstellen
    \item Hardwareplattformen
    \item Betriebssysteme der Devices
    \item Programmierung der Devices
    \item Kommunikationsprotokolle
    \item Backendkomponenten
    \item Benutzerseitige Komponenten
\end{enumerate}

\bigskip
\teilaufgabe
Anforderungen an die Software eines eingebetteten Systems:

\begin{longtable}{|p{0.45\textwidth} c p{0.45\textwidth}|}
    \hline
    Sparsame Ressourcennutzung
    & \textit{vs.} &
    \textcolor{gray}{\sout{Großzügige Zuteilung der Systemressourcen}}
    \\

    \hline
    \textcolor{gray}{\sout{Unterstützung üblicher Consumer-Hardware}}
    & \textit{vs.} &
    Unterstützung sehr spezialisierter Hardware
    \\

    \hline
    \textcolor{gray}{\sout{Geringe bis mittlere Nutzungsdauer am Tag}}
    & \textit{vs.} &
    Störungsfreier Betrieb über lange Zeit
    \\

    \hline
    Softwareupdates während dem Betrieb
    & \textit{vs.} &
    \textcolor{gray}{\sout{Updates in der Fachwerkstatt des Herstellers}}
    \\

    \hline
    \textcolor{gray}{\sout{Maximal flexible Systemarchitektur}}
    & \textit{vs.} &
    Vollständig deterministisches Systemverhalten
    \\

    \hline
    \textcolor{gray}{\sout{Nahezu unbegrenztes Multi-Tasking}}
    & \textit{vs.} &
    Feste Anzahl gleichzeitig laufender Tasks
    \\

    \hline
    \textcolor{gray}{\sout{Meist geringe Echtzeitanforderungen}}
    & \textit{vs.} &
    Erhöhte oder gar harte Echtzeitanforderungen
    \\

    \hline
    Ausschalten durch Trennen vom Strom
    & \textit{vs.} &
    \textcolor{gray}{\sout{Geordnetes Herunterfahren des Systems}}
    \\

    \hline
\end{longtable}

%-------------------------------------------------------------------------------
\loesung{Rechnerarchitekturen im Vergleich}
%-------------------------------------------------------------------------------
\teilaufgabe
TODO: Verschiedene Prozessoren einordnen

\teilaufgabe
TODO: Fragen zu einer Hardwareskizze

%-------------------------------------------------------------------------------
\loesung{Grundlagen der Assemblerprogrammierung}
%-------------------------------------------------------------------------------
TODO

%-------------------------------------------------------------------------------
\loesung{Komplexe vs. reduzierte Befehlssätze}
%-------------------------------------------------------------------------------
TODO

%-------------------------------------------------------------------------------
\loesung{Typischer Systemaufbau eingebetteter Systeme}
%-------------------------------------------------------------------------------
TODO
