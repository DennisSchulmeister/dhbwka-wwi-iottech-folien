\documentclass[12pt,epsfig]{article}

\usepackage[utf8x]{inputenc}
\usepackage[ngerman]{babel} %\usepackage{csquotes} \MakeOuterQuote{"}
\usepackage[a4paper]{geometry}
\usepackage{fancyhdr}

\usepackage{calc}
\usepackage{xspace} \newcommand{\latex}{\LaTeX\xspace} \newcommand{\tex}{\TeX\xspace}
\usepackage{tabularx}
\usepackage{tikz}
\usepackage{graphicx}
\usepackage{xcolor}
\usepackage{amsmath}
\usepackage{amssymb}
\usepackage{epsfig}
\usepackage{verbatim}
\usepackage{longtable}
\usepackage{wrapfig}
\usepackage{microtype}
\usepackage{fancyvrb}
\usepackage{totpages}
\usepackage{ulem}
%%%%%%%%%%%%%%%%%%%%%%%%%%%%%%%%%%%%%
%
% Umlaute f�r DOS und Windwos
%
\catcode`^^81=13 \def^^81{\"u}          %\catcode`^^81=12
\catcode`^^84=13 \def^^84{\"a}          %\catcode`^^84=12
\catcode`^^89=13 \def^^89{\"e}          %\catcode`^^89=12
\catcode`^^8e=13 \def^^8e{\"A}          %\catcode`^^8e=12
\catcode`^^94=13 \def^^94{\"o}          %\catcode`^^94=12
\catcode`^^99=13 \def^^99{\"O}          %\catcode`^^99=12
\catcode`^^9a=13 \def^^9a{\"U}          %\catcode`^^9a=12
\catcode`^^e1=13 \def^^e1{\ss{}}        %\catcode`^^e1=12
\catcode`^^fc=13 \def^^fc{\"u}          %\catcode`^^fc=12
\catcode`^^e4=13 \def^^e4{\"a}          %\catcode`^^e4=12
\catcode`^^89=13 \def^^89{\"e}          %\catcode`^^89=12
\catcode`^^c4=13 \def^^c4{\"A}          %\catcode`^^c4=12
\catcode`^^f6=13 \def^^f6{\"o}          %\catcode`^^f6=12
\catcode`^^d6=13 \def^^d6{\"O}          %\catcode`^^d6=12
\catcode`^^dc=13 \def^^dc{\"U}          %\catcode`^^dc=12
\catcode`^^df=13 \def^^df{\ss{}}        %\catcode`^^df=12
%
%%%%%%%%%%%%%%%%%%%%%%%%%%%%%%%%%%%%%

\def\qquad{\quad\quad}
\def\qqquad{\quad\quad\quad}

\def\define#1{\textbf{#1}}
\def\wichtig#1{\emph{#1}}

\newcommand{\bzw}{bzw.\xspace}
\renewcommand{\dh}{d.h.\xspace}
\newcommand{\Dh}{D.h.\xspace}
\newcommand{\eg}{e.g.\xspace}
\newcommand{\etc}{etc.\xspace}
\newcommand{\ggf}{ggf.\xspace}
\newcommand{\ie}{i.e.\xspace}
\newcommand{\oBdA}{o.B.d.A.\xspace}
\newcommand{\usw}{usw.\xspace}
\newcommand{\vgl}{vgl.\xspace}
\newcommand{\zB}{z.B.\xspace}
\newcommand{\ua}{u.a.\xspace}
\newcommand{\iA}{i.A.\xspace}
\newcommand{\ia}{i.a.\xspace}

\def\eps{\varepsilon}
\def\Z{\mathbb{Z}}
\def\N{\mathbb{N}}
\def\NullEins{\lbrack 0;1 \rbrack}
\def\NullEinsrechtsoffen{\lbrack 0;1 \lbrack}
\def\Q{\mathbb{Q}}
\def\R{\mathbb{R}}
\def\sig#1{\mathbb{#1}}
\def\mar#1{\mathbb{#1}}
\def\ld{\mathop{\hbox{\rm ld}}}
\let\setminus=\smallsetminus
\newcommand{\nfoldcomp}[2]{{#2}^{#1}} % das ist doch besser
%\newcommand{\nfoldcomp}[2]{{\vphantom{#2}}^{#1}\!#2} % das sieht noch suboptimal aus; aber
%                                   % amsmath's \sideset funktioniert nur f"ur
%                                   % gro"se Operatoren
\renewcommand{\O}{\mathrm{O}}
\def\maps[#1->#2]{{#2}^{#1}}

\def\za#1{{\tt\upshape\mdseries \uppercase{#1}}}
\def\prblm#1{{\tt\upshape\mdseries \uppercase{#1}}}

\def\computer#1{{\tt\upshape\mdseries #1}}
%\def\zs#1{\hbox{\tt\upshape\mdseries #1}}
\newcommand{\zs}[2][1.5em]{%
  \def\abczustandbreite{#1}%
  \if\abczustandbreite\relax
    \hbox{\texttt{#2}}%
  \else
    \hbox to #1{\hss\texttt{#2}\hss}%
  \fi
}
\newcommand{\sym}[1]{\ensuremath{\mathtt{#1}}}
\def\nostate{\sym{\_}}
\def\li{\hbox{\tt\upshape\mdseries L}}
\def\re{\hbox{\tt\upshape\mdseries R}}
\def\blank{\Box}
\def\Blank{\square}
\def\Black{\blacksquare}
\def\Wahl{\hbox to 0pt{$\Blank$\hss}\raise 1.0pt\hbox{$\times$}}
\def\jvn#1#2{\ensuremath{\mathbf{#1}_{#2}}}

\def\dontcare{\hbox{$\times$}}
\def\cod{\mathop{\hbox{\upshape\mdseries cod}}}
\def\wertber{\mathop{\hbox{\upshape\mdseries im}}}
\def\defber{\mathop{\hbox{\upshape\mdseries def}}}
\let\Lra\Longrightarrow

\def\0{\mathchoice
  {\hbox{\tt\upshape\mdseries 0}}
  {\hbox{\tt\upshape\mdseries 0}}
  {\hbox{\scriptsize\tt\upshape\mdseries 0}}
  {\hbox{\tiny\tt\upshape\mdseries 0}}}
\def\1{\mathchoice
  {\hbox{\tt\upshape\mdseries 1}}
  {\hbox{\tt\upshape\mdseries 1}}
  {\hbox{\scriptsize\tt\upshape\mdseries 1}}
  {\hbox{\tiny\tt\upshape\mdseries 1}}}
\def\2{\mathchoice
  {\hbox{\tt\upshape\mdseries 2}}
  {\hbox{\tt\upshape\mdseries 2}}
  {\hbox{\scriptsize\tt\upshape\mdseries 2}}
  {\hbox{\tiny\tt\upshape\mdseries 2}}}
\let\hashorig\#
\def\#{\mathchoice
  {\hbox{\tt\upshape\mdseries \hashorig}}
  {\hbox{\tt\upshape\mdseries \hashorig}}
  {\hbox{\scriptsize\tt\upshape\mdseries \hashorig}}
  {\hbox{\tiny\tt\upshape\mdseries \hashorig}}}
\newcommand{\ruhe}{%
  \hbox{%
    \kern 0.15ex
    \vrule height0.4ex width 0.15ex
    \vrule width 1ex height 0.15ex
    \vrule height0.4ex width 0.15ex
    \kern 0.15ex
}}

% fuer hical Programm Teile
\def\key#1{{\bf #1}}
\def\var#1{{\it #1}}

% fuer Zeilennummern in RZD
\newcounter{hugo}
\def\stephugo{\arabic{hugo}\addtocounter{hugo}{1}}

% PCX-Grafiken einbinden
%
% Argumente: #1: Breite
%            #2: H�he
%            #3: Dateiname ohne Extension
%
\newcommand{\pcxbild}[3]{%
 \unitlength1cm% 
 \begin{picture}(#1,#2)
   \put(0,#2){\special{em:graph #3.pcx}}
 \end{picture}}

%-----------------------------------------------------------------------------
% Definiere Befehl zur Erzeugung des Schluss-Spruches
%
\newcommand{\jokebox}[1]{%
     \begin{center}%
       \fbox{\fbox{~~\parbox{0.8\textwidth}{\parindent0em\sf{}
                                            ~\newline
                                            #1\newline
                                            ~
                                           }~~}}%
     \end{center}}
%
%-----------------------------------------------------------------------------




\usepackage{tikz}
\usetikzlibrary{calc}
\usepackage{color}
\definecolor{gray}{gray}{0.5}

\topmargin-2.25cm
\oddsidemargin-0.5cm
\evensidemargin-0.5cm
\headsep1cm
\textheight24cm
\textwidth17cm
\setlength{\parskip}{0.75ex plus 0.25ex minus 0.25ex}
\setlength{\parindent}{0.0cm}
\setlength{\abovecaptionskip}{0pt}
\setlength{\belowcaptionskip}{0pt}
\renewcommand{\rmdefault}{cmss}

% Abstände vor und nach Tabellen
\setlength{\LTpre}{1em}
\setlength{\LTpost}{1em}
\setlength{\textfloatsep}{1em}

\clubpenalty = 10000
\widowpenalty = 10000
\displaywidowpenalty = 10000

% Kompatibilität zu den Titelmakros von Beamer
\newcommand{\ubTitle}{}
\renewcommand{\title}[1]{\renewcommand{\ubTitle}{#1}}

\newcommand{\ubSubtitle}{}
\newcommand{\subtitle}[1]{\renewcommand{\ubSubtitle}{#1}}

\newcommand{\institute}[1]{}

\newcommand{\ubInstitute}{???}
\newcommand{\ubModule}{???}
\newcommand{\ubType}{???}
\newcommand{\seitenzahl}{\ref{TotPages}}
\newcommand{\kreuz}{}

\pagestyle{fancy}
\renewcommand{\headrulewidth}{0em}
\renewcommand{\footrulewidth}{0em}
\setlength{\headheight}{5em}
\fancyhead[L]{
    \bf
    \ubTitle
    \\
    \ubSubtitle
}
\fancyhead[R]{
    \includegraphics[width=3.25cm]{include/dhbwLogo.pdf}
    \hskip 0.5em
    \begin{minipage}[b]{4.455cm}
        \raggedright \bf
        {\scriptsize \ubInstitute} \\[-0.85ex]
        {\scriptsize \ubModule} \\[+1.2ex]
        {\normalsize Seite \arabic{page} von \seitenzahl}
    \end{minipage}
}
\fancyfoot[L,C,R]{}

\newcommand{\rechts}{\,\begin{math}\Rightarrow\end{math}\,}
\newcommand{\CodeTitel}[1]{\textbf{#1:}\vspace{-0.4cm}}

\newcommand{\AufgabenHeader}{
  \begin{center}
    {\large --- \ubType: \ubSubtitle\, ---}\\[4ex]
  \end{center}%
}

\newcommand{\LoesungHeader}{
  \begin{center}
    {\large --- Lösungen: \ubSubtitle\, ---}\\[4ex]
  \end{center}%
}

\newcounter{Aufgabe}
\setcounter{Aufgabe}{0}
\newcommand{\aufgabe}[1]{%
    \addtocounter{Aufgabe}{1}%
    \setcounter{Teilaufgabe}{0}%
    \subsection*{%
        Aufgabe \theAufgabe\hfill\textmd{#1}%
    }%
    \vspace{-0.5ex}%
}

\newcommand{\projekt}[2]{%
    \addtocounter{Aufgabe}{1}%
    \setcounter{Teilaufgabe}{0}%
    \subsection*{%
        Aufgabe \theAufgabe:~\textmd{#1} \hfill {\normalsize\bf (#2)}%
    }%
    \vspace{-0.5ex}%
}

\newcounter{Loesung}
\setcounter{Loesung}{0}
\newcommand{\loesung}[1]{%
    \addtocounter{Loesung}{1}%
    \setcounter{Teilaufgabe}{0}%
    \subsection*{%
        Lösung zur Aufgabe \theLoesung\hfill\textmd{#1}%
    }%
    \vspace{-0.5ex}%
}

\newcounter{Teilaufgabe}
\setcounter{Teilaufgabe}{0}
\newcommand{\teilaufgabe}{%
    \addtocounter{Teilaufgabe}{1}%
    \alph{Teilaufgabe}) %
}

\newcommand{\abgesetzt}[1]{\begin{center}#1\end{center}}
%%\newcommand{\texttt}[1]{{\tt #1}}
\newcommand{\und}{\,\wedge\,}
\newcommand{\oder}{\,\vee\,}

%=============================================================================
%  This is a modified MATHSING.STY 1.1 !!!
%  The following lines define a new environment 'directlisting'
%
%         \begin{directlisting}
%               ...
%         \end{directlisting}
%
%  that prints listings using \footnotesize and takes care to reset
%  the \baselineskip. The macro definition is based on ALLTT.STY that
%  allows various TEX commands to be given within the environment
%  (e.g. '\input', '\index' or '\it'). '%' has been retained as a special
%  character within 'listing', however, to avoid unwanted line breaks.
%=============================================================================
\def\docspecials{\do\ \do\$\do\&\do\\%
  \do\#\do\^\do\^^K\do\_\do\^^A\do\~\do\{\do\}\do\"}

\newdimen\oldbaselineskip
\makeatletter
\def\directlisting{%
  \par\noindent\oldbaselineskip=\baselineskip%
  \list{}{\leftmargin3mm\labelwidth0cm\listparindent0cm\itemindent0cm%
  \parsep=0.125\parskip}
%%%%  \def\baselinestretch{0.85}\footnotesize\tt%   <--  normalerweise
  \def\baselinestretch{0.85}\tt%                    <--  f�r �bungsbl�tter
  \item[]\if@minipage\else\vskip\parskip\fi%
  \leftskip\@totalleftmargin\rightskip\z@%
  \parindent\z@\parfillskip\@flushglue%
%%\parskip\z@%
  \@tempswafalse \def\par{\if@tempswa\hbox{}\fi\@tempswatrue\@@par}%
  \obeylines \Tn \catcode``=13 \@noligs \let\do\@makeother \docspecials%
  \frenchspacing\@vobeyspaces}
\makeatother

\def\enddirectlisting{\endlist\baselineskip=\oldbaselineskip}

%=============================================================================
% \directlisting is a command for directly typing program parts as short
% examples of coding.
%=============================================================================
\newcommand{\listing}[1]{%
  \begin{directlisting}\par\input{#1}\end{directlisting}}
%=============================================================================

\newcommand{\Tn}{}       % \tt for listings
\newcommand{\Tb}{}       % bold \tt for listings
\newcommand{\Ts}{}       % slanted \tt for listings



%-------------------------------------------------------------------------------
% Karos zum Lösen der Aufgaben
%-------------------------------------------------------------------------------
\newcommand{\GridOverlay}{}
\newlength{\MyRemainingHeight}
\newlength{\MyShift}

\newcommand{\DrawGrid}[1]{
    % Variante mit Höhenangabe zur Platzierung mitten in der Seite
    \begin{tikzpicture}
        \draw[lightgray](0,0) grid [step=.5cm] (\linewidth,#1);
        \GridOverlay
    \end{tikzpicture}
}

\newcommand{\EmptyGridPage}{
    % Variante für eine leere Seite
    \begin{tikzpicture}
        \draw[lightgray](0,0) grid [step=.5cm] (\linewidth,\textheight);
        \GridOverlay
    \end{tikzpicture}

    \newpage
}
