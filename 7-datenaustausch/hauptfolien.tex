%%% Folie
\begin{frame}{Ausgangslage}
    Hier die Ausgangslage/Motivation für das Kapitel beschreiben ...

    % UNGEFÄHRE AUSSAGE
    %
    % Dieselben Architekturmuster, die im Kleinen dazu genutzt werden können, einen
    % Quellcode in unabhängige Komponenten zu zerlegen, können im Großen auch genutzt
    % werden, um die Gesamtarchitektur eines IoT-Setups zu definieren. So können über
    % einen zentral gehosteten Message Broker mehrere Devices ganz einfach untereinander
    % Daten austauschen oder mit einem Cloud-Backend kommunizieren. Das mit Abstand
    % häufigste Protokoll hierfür ist MQTT, das vielen Programmiersprachen einfach
    % eingesetzt werden kann. Dieses Kapitel soll daher die Grundprinzipien asynchroner
    % Kommunikation mit dem Publish/Subscribe-Pattern vertiefen und zeigen, wie diese%
    % durch MQTT umgesetzt werden. Dabei soll auch gezeigt werden, wie MQTT zusammen mit
    % Python im Kontext eines typischen IoT-Cloudszenarios genutzt werden kann, um mehrere
    % Devices in einem Cloud-Dashboard zu überwachen.
\end{frame}

%%% Folie
\begin{frame}{Lernziele}
    \begin{itemize}
        \item Vor- und Nachteile asynchroner Kommunikationsmuster erklären können
        \item Skalierbare und ausfallsichere IoT-Architekturen entwerfen können
        \item Besonderheiten von MQTT bzgl. Publish/Subscribe kennnen
        \item MQTT-Clients in Python, Java und JavaScript entwickeln können
        \item Den Raspberry Pi in Python mit der Adafruit IO Cloud verbinden können
        \item Dashboards zur Überwachung und Steuerung mit Adafruit realisieren können
        %\item Einfache Workflows z.B: mit IFTTT oder Zapier realisieren können
    \end{itemize}
\end{frame}


%-------------------------------------------------------------------------------
\section{Abschnitt}
%-------------------------------------------------------------------------------

%%% Folie
\begin{frame}{Folie}
    TODO
\end{frame}

%%% Folie
\begin{frame}{Folie}
    TODO
\end{frame}

%%% Folie
\begin{frame}{Folie}
    TODO
\end{frame}

%-------------------------------------------------------------------------------
\section{Abschnitt}
%-------------------------------------------------------------------------------

%%% Folie
\begin{frame}{Folie}
    TODO
\end{frame}

%%% Folie
\begin{frame}{Folie}
    TODO
\end{frame}

